\documentclass[brazilian]{article}

\usepackage{babel}
\usepackage[utf8]{inputenc}
\usepackage[margin=1.35in]{geometry}
\usepackage[backend=biber,style=alphabetic]{biblatex}
\usepackage{mathtools,amssymb,amsthm}
\usepackage{indentfirst}
\usepackage{hyperref}
\usepackage{tikz-cd}
\usepackage{graphicx}
\usepackage[capitalize,noabbrev]{cleveref}

\usetikzlibrary{babel}

\addbibresource{bibliografia.bib}

\swapnumbers
\newtheorem{teo}{Teorema}[section]
\newtheorem{prop}[teo]{Proposição}
\newtheorem{lema}[teo]{Lema}
\newtheorem{corol}[teo]{Corolário}

\theoremstyle{definition}
\newtheorem{defin}[teo]{Definição}
\newtheorem{obs}[teo]{Observação}
\newtheorem{exem}[teo]{Exemplo}
\newtheorem{exer}[teo]{Exercício}

\swapnumbers

%Um ambiente do estilo Exercício, porém com numeração customizada.
\newtheorem{exernuminner}{Exercício}
\newenvironment{exernum}[1]{%
	\renewcommand\theexernuminner{#1}
	\exernuminner
}{\endexernuminner}

\DeclarePairedDelimiter{\abs}{\lvert}{\rvert}
\DeclarePairedDelimiter{\Abs}{\lVert}{\rVert}

\newcommand{\reg}[1]{#1_{\mathrm{reg}}}
\newcommand{\id}{\mathrm{id}}

\DeclareMathOperator{\colim}{colim}

\renewcommand{\qedsymbol}{$\blacksquare$}

\title{Notas sobre pushouts}
\author{Edmundo Martins}
\date{\today}
%%% Local Variables:
%%% mode: latex
%%% TeX-master: "main"
%%% End:


\begin{document}

\maketitle

\begin{abstract}
    Estas notas contêm uma exposição básica do conceito categórico de pushout.
    O foco principal de estudo são pushouts na categorias de espaços e funções contínuas, mas isso naturalmente requer o estudo de algumas propriedades gerais de pushouts, assim como o estudo de propriedades particulares de pushouts na categoria de conjuntos.
    Após isso, aplicamos os resultados obtidos para obter propriedades de algumas construções topológicas que aparecem com frequência na Topologia Algébrica.
\end{abstract}

\tableofcontents

\section{Pushouts gerais}

\begin{defin}\label{defin:span}
  Um \textbf{span} em uma categoria qualquer $\mathsf{C}$ é um diagrama da forma
  \begin{equation}
    \label{eq:diagrama_span}
    \begin{tikzcd}
      w
      \arrow[r, "\alpha"]
      \arrow[d, "\beta" swap]
      & x
      \\ y
    \end{tikzcd}
  \end{equation}
  onde $w$, $x$ e $y$ são objetos de $\mathsf{C}$, e $\alpha$ e $\beta$ são morfismos desta mesma categoria.
\end{defin}

\begin{defin}
  Dizemos que um span
  \begin{displaymath}
    \begin{tikzcd}
      w
      \arrow[r, "\alpha"]
      \arrow[d, "\beta" swap]
      & x
      \\ y
    \end{tikzcd}
  \end{displaymath}
  em uma categoria $\mathsf{C}$ \textbf{admite um pushout} se existem um objeto $z \in \mathsf{C}$ e morfismos $f: x \to z$ e $g: y \to z$ tais que $f   \circ \alpha = g \circ \beta$, e se a seguinte propriedade universal é satisfeita: dados um outro objeto $z' \in \mathsf{C}$ e outros dois morfismos $f': x \to z'$ e $g': y \to z'$ tais que $f' \circ \alpha = g' \circ \beta$, existe um \emph{único} morfismo $h: z \to z'$ tal que $h \circ f = f'$ e $h \circ g = g'$, ou seja, $h$  fatora os morfismos $f'$ e $g'$ por $f$ e $g$, respectivamente.
\end{defin}

No contexto da definição acima, dizemos também por vezes que a tripla $(z,f,g)$ é um \textbf{pushout} para o span inicial em questão.
Note que a condição $f \circ \alpha = g \circ \beta$ pode ser expressa graficamente por meio do quadrado comutativo abaixo.
\begin{equation}
  \label{eq:quadrado_pushout}
  \begin{tikzcd}
    w
    \arrow[r, "\alpha"]
    \arrow[d, "\beta" swap]
    & x
    \arrow[d, "f"]
    \\ y
    \arrow[r, "g" swap]
    & z
  \end{tikzcd}
\end{equation}
Ao invés de dizermos que a tripla $(z,f,g)$ é um pushout para o span inicial, também é bastante comum dizermos apenas que o diagrama comutativo acima é um pushout.

Também é possível - e altamente vantajoso - interpretar a propriedade universal que caracteriza o pushout em termos de diagramas.
Já vimos que um pushout $(z,f,g)$ determina um quadrado comutativo.
O outro objeto $z'$ juntamente com os morfismos $f': x \to z'$ e $g': y \to z'$ também determinam um quadrado comutativo, já que por hipótese vale a igualdade $f' \circ \alpha = g' \circ \beta$.
Os dois quadrados comutativos em questão podem ser combinados em um diagrama único como mostrado abaixo.
\begin{displaymath}
  \begin{tikzcd}
    w
    \arrow[r, "\alpha"]
    \arrow[d, "\beta" swap]
    & x
    \arrow[d, "f"]
    \arrow[rdd, bend left=15, "f'"]
    \\ y
    \arrow[r, "g" swap]
    \arrow[rrd, bend right=15, "g'" swap]
    & z
    \\ & & z'
  \end{tikzcd}
\end{displaymath}
O único morfismo $h: z \to z'$ induzido pela propriedade universal do pushout pode então ser representado pelo morfismo tracejado abaixo o qual faz o diagrama todo comutar.
\begin{displaymath}
  \begin{tikzcd}
    w
    \arrow[r, "\alpha"]
    \arrow[d, "\beta" swap]
    & x
    \arrow[d, "f"]
    \arrow[rdd, bend left=15, "f'"]
    \\ y
    \arrow[r, "g" swap]
    \arrow[rrd, bend right=15, "g'" swap]
    & z
    \arrow[rd, dashed, "h" description]
    \\ & & z'
  \end{tikzcd}
\end{displaymath}

Na Teoria de Categorias, objetos definidos por meio de propriedaddes universais satisfazem sempre algum tipo de unicidade a menos de isomorfismos, e isso não é diferente no caso de pushouts.

\begin{prop}\label{prop:unicidade_pushout}
    Suponha que os dois diagramas abaixo sejam pushouts em uma categoria $\mathsf{C}$.
    \begin{displaymath}
        \begin{tikzcd}
            w
            \arrow[r, "\alpha"]
            \arrow[d, "\beta" swap]
            & x
            \arrow[d, "f"]
            \\ y
            \arrow[r, "g" swap]
            & z
        \end{tikzcd}
        \qquad
        \begin{tikzcd}
            w
            \arrow[r, "\alpha"]
            \arrow[d, "\beta" swap]
            & x
            \arrow[d, "f'"]
            \\ y
            \arrow[r, "g'" swap]
            & z'
        \end{tikzcd}
    \end{displaymath}
    Então existe um \emph{único} isomorfismo $\theta: z \to z'$ satisfazendo as igualdades $\theta \circ f = f'$ e $\theta \circ g = g'$.
    Em outras palavras, quando um span admite um pushout, este é único a menos de isomorfismos.
\end{prop}

\begin{proof}
    Começamos com uma observação simples mas de extrema importância.
    Note que o morfismo $\id_z: z \to z$ faz o diagrama abaixo comutar.
    \begin{displaymath}
        \begin{tikzcd}
            w
            \arrow[r, "\alpha"]
            \arrow[d, "\beta" swap]
            & x
            \arrow[d, "f"]
            \arrow[rdd, bend left=15, "f"]
            \\ y
            \arrow[r, "g" swap]
            \arrow[rrd, bend right=15, "g" swap]
            & z
            \arrow[rd, "\id_z" description]
            \\ & & z
        \end{tikzcd}
    \end{displaymath}
    A propriedade universal do pushout no entanto garante que $\id_z$ é na verdade o \emph{único} morfismo do tipo $z \to z$ que faz tal diagrama comutar.
    Assim, se $\varphi: z \to z$ é um morfismo tal que $\varphi \circ f = f$ e $\varphi \circ g = g$, então necessariamente $\varphi = \id_z$.

    Vamos agora obter o isomorfismo em questão.
    A propriedade universal do pushout garante a existência de um único morfismo $\theta: z \to z'$ fazendo o diagrama abaixo comutar.
    \begin{displaymath}
        \begin{tikzcd}
            w
            \arrow[r, "\alpha"]
            \arrow[d, "\beta" swap]
            & x
            \arrow[d, "g"]
            \arrow[rdd, bend left=15, "g'"]
            \\ y
            \arrow[r, "f" swap]
            \arrow[rrd, bend right=15, "f'" swap]
            & z
            \arrow[rd, dashed, "\theta" description]
            \\ & & z'
        \end{tikzcd}
    \end{displaymath}
    
    Veja que $\theta$ satisfaz as duas igualdades impostas pelo enunciado.
    Resta então mostrarmos que $\theta$ é na verdade um isomorfismo, e para isso vamos exibir explicitamente o morfismo inverso.
    Usando novamente a propriedade universal do pushout obtemos o único morfismo $\theta': z' \to z$ fazendo o diagram abaixo comutar.
    \begin{displaymath}
        \begin{tikzcd}
            w
            \arrow[r, "\alpha"]
            \arrow[d, "\beta" swap]
            & x
            \arrow[d, "g'"]
            \arrow[rdd, bend left=15, "g"]
            \\ y
            \arrow[r, "f'" swap]
            \arrow[rrd, bend right=15, "f" swap]
            & z'
            \arrow[rd, dashed, "\theta'" description]
            \\ & & z
        \end{tikzcd}
    \end{displaymath}

    Vamos mostrar que $\theta$ e $\theta'$ são inversos.
    O truque para mostrarmos a igualdade $\theta' \circ \theta = \id_z$ é usarmos a caracterização do morfismo idêntico $\id_z$ discutida no primeiro parágrafo.
    Por um lado,
    \begin{displaymath}
        \theta' \circ \theta \circ f
        = \theta' \circ f' = f,
    \end{displaymath}
    e por outro,
    \begin{displaymath}
        \theta' \circ \theta \circ g
        = \theta' \circ g'
        = g.
    \end{displaymath}S
    Segue então da conclusão do primeiro parágrafo que $\theta' \circ \theta = \id_z$.
    O morfismo idêntico $\id_{z'}$ possui uma caracterização ànlaoga àquela do morfismo $\id_z$: se $\psi: z' \to z'$ satisfaz $\psi \circ f' = f'$ e $\psi \circ g' = g'$, então necessariamente $\psi = \id_{z'}$.
    Assim, as sequências de igualdades
    \begin{displaymath}
        \theta \circ \theta' \circ f'
        = \theta \circ f
        = f'
    \end{displaymath}
    e
    \begin{displaymath}
        \theta \circ \theta' \circ g'
        = \theta \circ g
        = g'
    \end{displaymath}
    implicam a igualdade $\theta \circ \theta' = \id_{z'}$ desejada.
\end{proof}

Veremos diversos exemplos concretos de pushouts ao longo do texto, mas desde já podemos estudar um exemplo geral que será posteriormente especializado para diferentes contextos.

\begin{exem}
    Seja $\mathsf{C}$ uma categoria contendo um objeto inicial $0 \in \mathsf{C}$.
    Dados dois objetos quaisquer $x,\, y \in \mathsf{C}$, temos os morfismos únicos $!_x: 0 \to x$ e $!_y: 0 \to y$ associados ao objeto inicial em questão.
    Considere o span abaixo,
    \begin{displaymath}
        \begin{tikzcd}
            0
            \arrow[r, "!_x"]
            \arrow[d, "!_y" swap]
            & x
            \\ y
        \end{tikzcd}
    \end{displaymath}
    e suponha que ele admita um pushout como mostrado abaixo.
    \begin{displaymath}
        \begin{tikzcd}
            0
            \arrow[r, "!_x"]
            \arrow[d, "!_y" swap]
            & x
            \arrow[d, "i"]
            \\ y
            \arrow[r, "j" swap]
            & z
        \end{tikzcd}
    \end{displaymath}

    Afirmamos que nesse caso $z$ é um \emph{coproduto} dos objetos $x$ e $y$, sendo $i$ e $j$ as injeções canônicas.
    Podemos mostrar isso simplesmente verificando que a propriedade universal do coproduto é satisfeita.
    Seja $z'$ um outro objeto de $\mathsf{C}$ equipado com morfismos $i': x \to z'$ e $j': y \to z'$.
    Como existe um único morfismo do tipo $0 \to z'$, os morfismos compostos $i' \circ !_x,\, j' \circ j' \circ !_y: 0 \to z'$ são iguais, portanto a propriedade universal do pushout garante a existência de um único morfismo $h: z \to z''$ fazendo comutar o diagram abaixo.
    \begin{displaymath}
        \begin{tikzcd}
            0
            \arrow[r, "!_x"]
            \arrow[d, "!_y" swap]
            & x
            \arrow[d, "i"]
            \arrow[rdd, bend left=15, "i'"]
            \\ y
            \arrow[r, "j" swap]
            \arrow[rrd, bend right=15, "j'" swap]
            & z
            \arrow[rd, dashed, "h" description]
            \\ & & z'
        \end{tikzcd}
    \end{displaymath}
    Em outras palavras, os morfismos $i'$ e $j'$ podem ser unicamente fatorados pelos morfismos $i$ e $j$, respectivamente, o que significa precisamente que a tripla $(z,i,j)$ define um ocproduto de  $x$ e $y$.

    Esse exemplo mostra que, em uma categoria que contendo um objeto inicial, coprodutos podem ser vistos como pushouts.
    Dito de outra forma, um pushout pode ser visto como uma generalização de um coproduto.
    Mais adiante, veremos inclusive que, em categorias suficientemente ricas, pushouts podem ser construídos \emph{a partir} de coprodutos usando um certo tipo de quociente.
\end{exem}

\subsection{Propriedades gerais}

O objetivo dessa seção é demonstrar algumas propriedades de pushouts que são válidas em qualquer categoria.
Posteriormente, teremos a oportunidade de aplicar tais resultados quando estudarmos pushouts em algumas categorias particulares.

O primeiro resultado oferece uma receita para a construção de pushouts em termos de  coprodutos e coequalizadores.

\begin{teo}
   Considere o diagrama abaixo em uma categoria $\mathsf{C}$.
   \begin{displaymath}
    \begin{tikzcd}
        w
        \arrow[r, "\alpha"]
        \arrow[d, "\beta" swap]
        & x
        \\ y
    \end{tikzcd}
   \end{displaymath}
   Suponha que os objetos $x$ e $y$ admitam um coproduto $x+y$ com injeções canônicas $i_1: x \to x + y$ e $i_2: y \to x+y$.
   Nesse contexto, o diagrama acima admite um pushout se, e somente se, o par de morfismos paralelos
   \begin{displaymath}
    \begin{tikzcd}
        w
        \arrow[r, shift left=1.25, "i_1 \circ \alpha"]
        \arrow[r, shift right=1.25, "i_2 \circ \beta" swap]
        & x+y
    \end{tikzcd}
   \end{displaymath}
   admite um coequalizador.
\end{teo}

\begin{proof}
    Suponha inicialmente que exista um pushout como abaixo.
    \begin{displaymath}
        \begin{tikzcd}
            w
            \arrow[r, "\alpha"]
            \arrow[d, "\beta" swap]
            & x
            \arrow[d, "f"]
            \\ y
            \arrow[r, "g" swap]
            & z
        \end{tikzcd}
    \end{displaymath}
    Os morfismos $f$ e $g$ induzem um morfismo $\langle f,g \rangle: x +y \to z$ por meio da propriedade universal do coproduto.
    Vamos mostrar que esse morfismo induzido é o coequalizador desejado.
    Note, em primeiro lugar, que ele coequaliza os morfismos necessários, pois
    \begin{align*}
        \langle f,g \rangle \circ i_1 \circ \alpha
        & = f \circ \alpha \\
        & = g \circ \beta \\
        & = \langle f,g \rangle \circ i_2 \circ \beta.
    \end{align*}
    Em outras palavras, temos o diagrama comutativo abaixo.
    \begin{displaymath}
        \begin{tikzcd}
            w
            \arrow[r, shift left=1.25, "i_1 \circ \alpha"]
            \arrow[r, shift right=1.25, "i_2 \circ \beta" swap]
            & x+y
            \arrow[r, "{\langle f,g \rangle}"]
            & z
        \end{tikzcd}
    \end{displaymath}

    Agora, temos que mostrar que, se $p: x+y \to z'$ é outro morfismo coequalizando $i_1 \circ \alpha$ e $i_2 \circ \beta$, então existe um único morfismo $h: z \to z'$ tal que $h \circ \langle f,g  \rangle = p$, o que pode ser expresso pela comutatividade do diagrama abaixo.
    \begin{displaymath}
        \begin{tikzcd}
            w
            \arrow[r, shift left=1.25, "i_1 \circ \alpha"]
            \arrow[r, shift right=1.25, "i_2 \circ \beta" swap]
            & x+y
            \arrow[r, "{\langle f,g \rangle}"]
            \arrow[rd, "p" swap]
            & z
            \arrow[d, dashed, "h"]
            \\ & & z'
        \end{tikzcd}
    \end{displaymath}
    Note que, como os morfismos $p$ e $h \circ \langle f,g \rangle$ têm $x+y$ como domínio, a propriedade universal do coproduto garante a seguinte equivalência:
    \begin{displaymath}
        p = h \circ \langle f,g \rangle
        \iff
        \begin{cases}
            p \circ i_1 = h \circ \langle f,g \rangle \circ i_1 \\
            p \circ i_2 = h \circ \langle f,g \rangle \circ i_2.
        \end{cases}
    \end{displaymath}
    Usando as igualdades que definem o morfismo induzido $\langle f,g \rangle$ podemos reescrever a esquivalência acima  da seguinte maneira:
    \begin{displaymath}
        p = h \circ \langle f,g \rangle \iff
        \begin{cases}
            p \circ i_1 = h \circ f \\
            p \circ i_2 = h \circ g.
        \end{cases}
    \end{displaymath}
    
    Vamos então construir o morfismo desejado.
    Como $p$ satisfaz a igualdade $p \circ i_1 \circ \alpha = p \circ i_2 \circ \beta$ por hipótese, a propriedade universal do pushout garante a existência de um único morfismo $h: z \to z'$ fazendo o diagrama abaixo comutar.
    \begin{displaymath}
        \begin{tikzcd}
            w
            \arrow[r, "\alpha"]
            \arrow[d, "\beta" swap]
            & x
            \arrow[d, "f"]
            \arrow[rdd, bend left=15, "p \circ i_1"]
            \\ y
            \arrow[r, "g" swap]
            \arrow[rrd, bend right=15, "p \circ i_2" swap]
            & z
            \arrow[rd, dashed, "h" description]
            \\ & & z'
        \end{tikzcd}
    \end{displaymath}
    É imediato da comutatividade acima que esse único morfismo $h$ é exatamente o que estamos procurando.

    Reciprocamente, suponha agora que os morfismos $i_1 \circ \alpha,\, i_2 \circ \beta: w \to x+y$ admitam um coequalizador $p: x+y \to z$.
    Vamos mostrar então que o diagrama abaixo é um pushout.
    \begin{displaymath}
        \begin{tikzcd}
            w
            \arrow[r, "\alpha"]
            \arrow[d, "\beta" swap]
            & x
            \arrow[d, "p \circ i_1"]
            \\ y
            \arrow[r, "p \circ i_2" swap]
            & z
        \end{tikzcd}
    \end{displaymath}
    Seja $z'$ um outro objeto juntamente com morfismos $j_1: x \to z'$ e $j_2: y \to z'$ satisfazendo a igualdade $j_1 \circ \alpha = j_2 \circ \beta$.
    Veja então que
    \begin{displaymath}
      \langle j_1,j_2 \rangle \circ i_1 \circ \alpha = j_1 \circ \alpha = j_2 \circ \beta = \langle j_1,j_2 \rangle \circ i_2 \circ \beta,
    \end{displaymath}
    ou seja, o morfismo $\langle j_1, j_2 \rangle: x+y \to z'$ induzido pela propriedade universal do coproduto coequaliza os morfismos $i_1 \circ \alpha$ e $i_2 \circ \beta$.
    A propriedade universal do coequalizador garante então a existência de um único morfismo $h: z \to z'$ fazendo o diagram abaixo comutar.
    \begin{displaymath}
        \begin{tikzcd}
            w
            \arrow[r, shift left=1.25, "i_1 \circ \alpha"]
            \arrow[r, shift right=1.25, "i_2 \circ \beta" swap]
            & x+y
            \arrow[r, "p"]
            \arrow[rd, "{\langle j_1,j_2 \rangle}" swap]
            & z
            \arrow[d, dashed, "h"]
            \\ & & z'
        \end{tikzcd}
    \end{displaymath}
    Vemos então que por um lado
    \begin{displaymath}
        h \circ p \circ i_1
        = \langle j_1,j_2 \rangle \circ i_1
        = j_1,
    \end{displaymath}
    enquanto por outro
    \begin{displaymath}
      h \circ p \circ i_2
      = \langle j_1,j_2 \rangle \circ i_2
      = j_2;
    \end{displaymath}
    mostrando que $h$ satisfaz as condições de comutatividade necessárias.

    A fim de concluirmos que a tripla $(z,p \circ i_1, p \circ i_2)$ é o pushout desejado, resta apenas mostrarmos que o morfismo $h$ obtido acima é na verdade o único satisfazendo as condições necessárias.
    Suponha então que $h': z \to z'$ também seja tal que $h' \circ p \circ i_1 = j_1$ e $h' \circ p \circ i_2 = j_2$.
    Segue da unicidade na propriedade universal do coproduto que $h' \circ p = \langle j_1,j_2 \rangle$, mas como $h$ foi obtido da propriedade universal do coequalizador, segue dessa última igualdade que $h'=h$.
\end{proof}

Nosso próximo resultado mostra quais propriedades de um morfismo são preservadas em um diagrama de pushout.

\begin{teo}\label{teo:props_preservadas_pushout}
    Considere o diagrama de pushout abaixo em uma categoria qualquer $\mathsf{C}$.
    \begin{displaymath}
        \begin{tikzcd}
            w
            \arrow[r, "\alpha"]
            \arrow[d, "\beta" swap]
            & x
            \arrow[d, "g"]
            \\ y
            \arrow[r, "f" swap]
            & z
        \end{tikzcd}
    \end{displaymath}
    As seguintes afirmações são verdadeiras:
    \begin{enumerate}
        \item[(a)] se $\beta$ é um epimorfismo, então $g$ é um epimorfismo;

        \item[(b)] se $\beta$ é um isomorfismo, então $g$ é um isomorfismo.
    \end{enumerate}
\end{teo}

\begin{proof}
    (a) Dados dois morfismos $\varphi,\, \psi: z \to z'$ tais que $\varphi \circ g = \psi \circ g$, precismos mostrar que $\varphi = \psi$.
    Note que, como $(\varphi \circ g) \circ \alpha = (\varphi \circ f) \circ \beta$, a propriedade universal do pushout garante que $\varphi$ é o único morfismo do tipo $z \to z'$ que faz o diagrama abaixo comutar.
    \begin{equation}\label{eq:pushout_preserva_epi_1}
        \begin{tikzcd}
            w
            \arrow[r, "\alpha"]
            \arrow[d, "\beta" swap]
            & x
            \arrow[d, "g"]
            \arrow[rdd, bend left=15, "\varphi \circ g"]
            \\ y
            \arrow[r, "f" swap]
            \arrow[rrd, bend right=15, "\varphi \circ f" swap]
            & z
            \arrow[rd, "\varphi" description]
            \\ & & z'
        \end{tikzcd}
    \end{equation}
    Analogamente, o morfismo $\psi$ é o único que faz o diagrama abaixo comutar.
    \begin{equation}\label{eq:pushout_preserva_epi_2}
        \begin{tikzcd}
            w
            \arrow[r, "\alpha"]
            \arrow[d, "\beta" swap]
            & x
            \arrow[d, "g"]
            \arrow[rdd, bend left=15, "\psi \circ g"]
            \\ y
            \arrow[r, "f" swap]
            \arrow[rrd, bend right=15, "\psi \circ f" swap]
            & z
            \arrow[rd, "\psi" description]
            \\ & & z'
        \end{tikzcd}
    \end{equation}

    Perceba que os diagramas \eqref{eq:pushout_preserva_epi_1} e \eqref{eq:pushout_preserva_epi_2} não são tão diferentes assim, já que por hipótese os morfismos $\varphi \circ g$ e $\psi \circ g$ são iguais.
    Se mostrarmos que os morfismos $\varphi \circ f$ e $\psi \circ g$ também são iguais, então $\varphi$ e $\psi$ serão ambos induzidos pela propriedade universal de um mesmo diagrama de pushout, o que implica a igualdade $\varphi = \psi$.
    Usando a comutatividade do diagrama de pushout do enunciado vemos que
    \begin{align*}
        \varphi \circ f \circ \beta
        & = \varphi \circ g \circ \alpha \\
        & = \psi \circ g \circ \alpha \\
        & = \psi \circ f \circ \beta,
    \end{align*}
    e sendo $\beta$ um epimorfismo por hipótese, segue da igualdade acima que $\varphi \circ f = \psi \circ f$ como desejávamos mostrar.

    \smallskip
    (b) Considere o morfismo inverso $\beta^{-1}: y \to w$.
    Como $(\alpha \circ \beta^{-1}) \circ \beta = \id_x \circ \alpha$, o quadrado externo no diagram abaixo é comutativo, e a propriedade universal do pushout implica então a existência de um único morfismo $h: z \to x$ fazendo comutar o diagrama todo.
    \begin{displaymath}
        \begin{tikzcd}
            w
            \arrow[r, "\alpha"]
            \arrow[d, "\beta" swap]
            & x
            \arrow[d, "g"]
            \arrow[rdd, bend left=15, "\id_x"]
            \\ y
            \arrow[r, "f" swap]
            \arrow[rrd, bend right=15, "\alpha \circ \beta^{-1}" swap]
            & z
            \arrow[rd, dashed, "h" description]
            \\ & & x
        \end{tikzcd}
    \end{displaymath}

    Vamos mostrar que $h$ é o morfismo inverso de $g$.
    Perceba que a igualdade $h \circ g = \id_x$ segue imediatamente da comutatividade do diagrama acima, restando apenas mostrarmos a igualdades $g \circ h = \id_z$.
    Lembre agora que, como discutido no primeiro parágrafo da demonstração da \cref{prop:unicidade_pushout}, o morfismo idêntico $\id_z$ é o único do tipo $z \to z$ que satisfaz as igualdades $\id_z \circ f = f$ e $\id_z \circ g = g$.
    Note, entretanto, que por um lado
    \begin{align*}
        g \circ h \circ f
        & = g \circ \alpha \circ \beta^{-1} \\
        & = f \circ \beta \circ \beta^{-1} \\
        & = f \circ \id_y \\
        & = f,
    \end{align*}
    e por outro
    \begin{displaymath}
        g \circ h \circ g
        = g \circ \id_x
        = g.
    \end{displaymath}
    Segue então dessas duas igualdades e da observação anterior que $g \circ h = \id_z$.
 \end{proof}

 O próximo resultado mostra como pushouts podem ser combinados para obtermos novos pushouts.

 \begin{teo}[Lei de colagem para pushouts]
   Considere o diagrama comutativo abaixo em uma categoria qualquer $\mathsf{C}$,
   \begin{equation}\label{eq:colagem-pushout-diag-1}
     \begin{tikzcd}
       u
       \arrow[r, "\alpha"]
       \arrow[d, "f" swap]
       & v
       \arrow[r, "\beta"]
       \arrow[d, "g" swap]
       & w
       \arrow[d, "h"]
       \\ x
       \arrow[r, "\gamma" swap]
       & y
       \arrow[r, "\delta" swap]
       & z
     \end{tikzcd}
   \end{equation}
   e considere também o diagrama abaixo obtido pela ``colagem'' dos dois quadrados comutativos acima.
   \begin{equation}\label{eq:colagem-pushout-diag-2}
     \begin{tikzcd}
       u
       \arrow[r, "\beta \circ \alpha"]
       \arrow[d, "f" swap]
       & w
       \arrow[d, "h"]
       \\ x
       \arrow[r, "\delta \circ \gamma" swap]
       & z
     \end{tikzcd}
   \end{equation}

   Se no diagrama \eqref{eq:colagem-pushout-diag-1} o quadrado da esquerda é um pushout, então as seguintes afirmações são equivalentes:
   \begin{enumerate}
   \item o quadrado da direita no diagrama \eqref{eq:colagem-pushout-diag-2} é um pushout;
   \item o diagrama \eqref{eq:colagem-pushout-diag-2} é um pushout.
   \end{enumerate}
 \end{teo}

 \begin{proof}
   Suponhamos primeiro que o quadrado da direita no diagrama \eqref{eq:colagem-pushout-diag-1} seja um pushout.
   A fim de mostrarmos que o diagrama \eqref{eq:colagem-pushout-diag-2} é também um pushout, considere um outro objeto $z'$ juntamente com morfismos $\varphi: w \to z'$ e $\psi: x \to z'$ satisfazendo a igualdade $\varphi \circ \beta \circ \alpha = \psi \circ f$, como mostrado abaixo.
   \begin{displaymath}
     \begin{tikzcd}
        u
       \arrow[r, "\beta \circ \alpha"]
       \arrow[d, "f" swap]
       & w
       \arrow[d, "h"]
       \arrow[rdd, bend left=15, "\varphi"]
       \\ x
       \arrow[r, "\delta \circ \gamma" swap]
       \arrow[rrd, bend right=15, "\psi" swap]
       & z
       \\ & & z'
     \end{tikzcd}
   \end{displaymath}

   Note que a ``camada externa'' do diagrama abaixo também é comutativa, e como o quadrado esquerdo em \eqref{eq:colagem-pushout-diag-1} é um pushout, obtemos um único morfismo $\lambda: y \to z'$ fazendo comutar todo o diagram abaixo.
   \begin{displaymath}
     \begin{tikzcd}
       u
       \arrow[r, "\alpha"]
       \arrow[d, "f" swap]
       & v
       \arrow[d, "g"]
       \arrow[rdd, bend left=15, "\varphi \circ \beta"]
       \\ x
       \arrow[r, "\gamma" swap]
       \arrow[rrd, bend right=15, "\psi" swap]
       & y
       \arrow[rd, dashed, "\lambda" description]
       \\ & & z'
     \end{tikzcd}
   \end{displaymath}
   O morfismo $\lambda$ por sua vez faz comutar a camada externa do diagrama abaixo, e como o quadrado direito em \eqref{eq:colagem-pushout-diag-2} é também um pushout por hipótese, obtemos um único morfismo $\theta: z \to z'$ fazendo comutar todo o diagrama.
   \begin{displaymath}
     \begin{tikzcd}
       v
       \arrow[r, "\beta"]
       \arrow[d, "g" swap]
       & w
       \arrow[rdd, bend left=15, "\varphi"]
       \arrow[d, "h"]
       \\ y
       \arrow[r, "\delta" swap]
       \arrow[rrd, bend right=15, "\lambda" swap]
       & z
       \arrow[rd, dashed, "\theta" description]
       \\ & & z'
     \end{tikzcd}
   \end{displaymath}

   Veja que a igualdade $\theta \circ h = \varphi$ segue imediatamente da comutatividade acima.
   Além disso, também temos
   \begin{displaymath}
     \theta \circ \delta \circ \gamma
     = \lambda \circ \gamma
     = \psi,
   \end{displaymath}
   onde a primeira igualdade segue da definição de $\theta$, e a segunda segue da definição de $\lambda$.
   Isso significa que o morfismo $\theta$ satisfaz as condições de comutatividade necessárias, mas ainda temos que mostrar que ele é o único possuindo tais propriedades.
   Suponha então que $\theta': z \to z'$ seja outro morfismo satisfazendo as igualdades $\theta' \circ h = \varphi$ e $\theta' \circ \delta \circ \gamma = \psi$.
   Usando a forma como $\theta$ foi definido, é suficiente mostrarmos que $\theta'$ também satisfaz a igualdade $\theta' \circ \delta = \lambda$.
   O morfismo $\lambda$, por sua vez, é unicamente caracterizado pelas igualdades $\lambda \circ \gamma = \psi$ e $\lambda \circ g = \varphi \circ \beta$.
   Ora, por um lado $\theta'$ ja satisfaz a igualdade $\theta' \circ \delta \circ \gamma = \psi$ por hipótese, e por outro
   \begin{displaymath}
     \theta' \circ \delta \circ g = \theta' \circ h \circ \beta = \varphi \circ \beta,
   \end{displaymath}
   sendo que a primeira igualdade segue da comutatividade do quadrado direito em \eqref{eq:colagem-pushout-diag-1}, enquanto a segunda vale por hipótese.
   Deduzimos disso que $\theta' \circ \delta = \lambda$, o que juntamente com a igualdade $\theta' \circ h = \varphi$ implica a igualdade $\theta' = \theta$ desejada.
 \end{proof}

\section{As muitas vidas de um pushout}
\label{sec:vidas_pushout}

\subsection{Um (co?)epsilon de colimites}
\label{sec:revisao_colimites}

Esta subseção contém uma revisão muito breve da definição de colimite de um funtor.
O leitor que nunca tenha encontrado esse conceito provavelmente se beneficiará  mais de uma leitura do capítulo 3 do livro \cite{cat_riehl}.

A noção de colimite é uma das formas mais úteis de construirmos novos objetos em uma categoria.
Intuitivamente, um colimite representa a forma ``universal'' de mapearmos para ``fora'' de um diagram.
Existem diferentes maneiras de definir tal conceito, e aqui faremos isso usando a noção de funtor representável.

Sejam $\mathsf{J}$ uma categoria pequena e $\mathsf{C}$ uma categoria localmente pequena.
Um objeto qualquer $c \in \mathsf{C}$ determina um funtor $\Delta(c): \mathsf{J} \to \mathsf{C}$ definido da seguinte maneira: se $j \in \mathsf{J}$ é um objeto, então $\Delta(c)(j) \coloneqq c$, e se $\alpha: j_1 \to j_2$ é um morfismo em $\mathsf{J}$, então $\Delta(c)(\alpha) \coloneqq \id_c: \Delta(c)(j_1) \to \Delta(c)(j_2)$.
O funtor $\Delta(c)$ assim definido é chamado \textbf{funtor constante e igual a $c$}.

É natural nos indagarmos sobre a dependência dessa construção do objeto $c$ fixado incialmente.
Se $c' \in \mathsf{C}$ é outro objeto, e $f: c \to c'$ é um morfismo, então $f$ é um morfismo do tipo $\Delta(c)(j) \to \Delta_(c')(j)$ para todo $j \in \mathsf{J}$.
Além disso, se $\alpha: j_1 \to j_2$ é um morfismo em $\mathsf{J}$, temos as igualdades
\begin{displaymath}
  f \circ \Delta(c)(\alpha) = f \circ \id_c = f = \id_{c'} \circ f = \Delta(c')(\alpha) \circ f,
\end{displaymath}
o que pode ser representado pelo quadrado comutativo abaixo.
\begin{displaymath}
  \begin{tikzcd}[column sep=1.5cm]
    \Delta(c)(j_1) = c
    \arrow[r, "\Delta(c)(\alpha) = \id_c"]
    \arrow[d, "f" swap]
    & \Delta(c)(j_2) = c
    \arrow[d, "f"]
    \\ \Delta(c')(j_1) = c'
    \arrow[r, "\Delta(c')(\alpha) = \id_{c'}" swap]
    & \Delta(c')(j_2) = c'
  \end{tikzcd}
\end{displaymath}
Em outras palavras, a tupla de morfismos $(f: \Delta(c)(j) \to \Delta(c')(j))_{j \in \mathsf{J}}$ define uma transformação natural do tipo $\Delta(c) \Rightarrow \Delta(c')$, a qual denotaremos por $\Delta(f)$.

Até o momento, vimos que um objeto $c \in \mathsf{C}$ dá origem a um funtor constante $\Delta(c): \mathsf{J} \to \mathsf{C}$, e que um morfismo $f: c \to c'$ dá origem a uma transformação natural $\Delta(f): \Delta(c) \Rightarrow \Delta(c')$ entre os funtores constantes correspondentes.
Lembrando que funtores do tipo $\mathsf{J} \to \mathsf{C}$ e transformações naturais entre eles são, respectivamente, os objetos e morfismos da categoria de funtores $\mathsf{C^J}$, é de se esperar que a associação $c \mapsto \Delta(c)$ e $(f: c \to c') \mapsto (\Delta(f): \Delta(c) \Rightarrow \Delta(c'))$ defina ela própria um funtor do tipo $\mathsf{C} \to \mathsf{C^J}$.
De fato, se considerarmos o morfismo idêntico $\id_c: c \to c$, então a $j$-ésima componente da transformação natural $\Delta(\id_c): \Delta(c) \Rightarrow \Delta(c)$ é por definição igual a $\id_c$, portanto $\Delta(\id_c)$ é igual à transformação natural idêntica $\id_{\Delta(c)}$ do funtor $\Delta(c)$ para si mesmo.
Além disso, dados dois morfismos $f: c_1 \to c_2$ e $g: c_2 \to c_3$ na categoria $\mathsf{C}$, a $j$-ésima componente da transformação $\Delta(g \circ f): \Delta(c_1) \Rightarrow \Delta(c_3)$ é por definição igual $g \circ f$, ou seja, $\Delta(g \circ f)_j = \Delta(g)_j \circ \Delta(f)_j$; portanto $\Delta(g \circ f)$ é igual à composição de transformações naturais $\Delta(g) \circ \Delta(f)$.
O funtor $\mathsf{C} \to \mathsf{C^J}$ assim definido, o qual denotaremos por $\Delta$, é chamado de \textbf{funtor diagonal}.

Agora, considere um funtor $F: \mathsf{J} \to \mathsf{C}$.
No contexto do estudo de limites e colimites, é comum nos referirmos a funtor deste tipo como um \textbf{diagrama comutativo do tipo $\mathsf{J}$ na categoria $\mathsf{C}$}.
Essa nomenclatura faz sentido porque, especialmente no caso em que a categoria $\mathsf{J}$ é finita, podemos visualizar $F$ como um diagrama formado por objetos e morfismos de $\mathsf{C}$.
Por exemplo, se $\mathsf{J}$ é uma categoria contendo dois objetos $0$ e $1$ e um único morfismo não idêntico $\alpha: 0 \to 1$, então um funtor $F: \mathsf{J} \to \mathsf{C}$ é o mesmo que um diagrama em $\mathsf{C}$ da forma
\begin{displaymath}
  \begin{tikzcd}
    c_0
    \arrow[r, "f"]
    & c_1,
  \end{tikzcd}
\end{displaymath}
onde $c_0 \coloneqq F(0)$, $c_1 \coloneqq F(1)$ e $f \coloneqq F(\alpha)$.\footnote{Note que o funtor $F: \mathsf{J} \to \mathsf{C}$ fica totalmente determinado nesse caso pela escolha de um morfismo na categoria $\mathsf{C}$. Por esse motivo a categoria $\mathsf{J}$ usada nesse exemplo, a qual é mais comumente denotada por $[1]$, é chamada de \textbf{morfismo ambulante} ou \textbf{seta ambulante}. A caegoria de funtores $\mathsf{C}^{[1]}$ é então chamada de \textbf{categoria de morfismos de $\mathsf{C}$} ou \textbf{categoria de setas de $\mathsf{C}$}.}


\printbibliography

\end{document}
%%% Local Variables:
%%% mode: latex
%%% TeX-master: t
%%% End:

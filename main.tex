\documentclass[brazilian]{article}

\usepackage{babel}
\usepackage[utf8]{inputenc}
\usepackage[margin=1.35in]{geometry}
\usepackage[backend=biber,style=alphabetic]{biblatex}
\usepackage{mathtools,amssymb,amsthm}
\usepackage{indentfirst}
\usepackage{hyperref}
\usepackage{tikz-cd}
\usepackage{graphicx}
\usepackage[capitalize,noabbrev]{cleveref}

\usetikzlibrary{babel}

\addbibresource{bibliografia.bib}

\swapnumbers
\newtheorem{teo}{Teorema}[section]
\newtheorem{prop}[teo]{Proposição}
\newtheorem{lema}[teo]{Lema}
\newtheorem{corol}[teo]{Corolário}

\theoremstyle{definition}
\newtheorem{defin}[teo]{Definição}
\newtheorem{obs}[teo]{Observação}
\newtheorem{exem}[teo]{Exemplo}
\newtheorem{exer}[teo]{Exercício}

\swapnumbers

%Um ambiente do estilo Exercício, porém com numeração customizada.
\newtheorem{exernuminner}{Exercício}
\newenvironment{exernum}[1]{%
	\renewcommand\theexernuminner{#1}
	\exernuminner
}{\endexernuminner}

\DeclarePairedDelimiter{\abs}{\lvert}{\rvert}
\DeclarePairedDelimiter{\Abs}{\lVert}{\rVert}

\newcommand{\reg}[1]{#1_{\mathrm{reg}}}
\newcommand{\id}{\mathrm{id}}

\DeclareMathOperator{\colim}{colim}

\renewcommand{\qedsymbol}{$\blacksquare$}

\title{Notas sobre pushouts}
\author{Edmundo Martins}
\date{\today}
%%% Local Variables:
%%% mode: latex
%%% TeX-master: "main"
%%% End:


\begin{document}

\maketitle

\begin{abstract}
    Estas notas contêm uma exposição básica do conceito categórico de pushout.
    O foco principal de estudo são pushouts na categorias de espaços e funções contínuas, mas isso naturalmente requer o estudo de algumas propriedades gerais de pushouts, assim como o estudo de propriedades particulares de pushouts na categoria de conjuntos.
    Após isso, aplicamos os resultados obtidos para obter propriedades de algumas construções topológicas que aparecem com frequência na Topologia Algébrica.
\end{abstract}

\tableofcontents

\section{Pushouts gerais}

Seja $\mathsf{C}$ uma categoria arbitrária.
Dizemos que um quadrado comutativo em $\mathsf{C}$ da forma
\begin{equation}\label{eq:diagrama_pushout}
    \begin{tikzcd}
        w
        \arrow[r, "\alpha"]
        \arrow[d, "\beta" swap]
        & x
        \arrow[d, "g"]
        \\ y
        \arrow[r, "f" swap]
        & z
    \end{tikzcd}
\end{equation}
é um \textbf{pushout} se ele satisfaz a seguinte propriedade universal: se $z'$ é outro objeto, e $g': x \to z$ e $f': y \to z'$ são morfismos tais que $g' \circ \alpha' = f' \circ \beta'$, ou seja, tais que o diagrama abaixo seja comutativo;
\begin{displaymath}
    \begin{tikzcd}
        w
        \arrow[r, "\alpha"]
        \arrow[d, "\beta" swap]
        & x
        \arrow[d, "g"]
        \arrow[rdd, bend left=15, "g'"]
        \\ y
        \arrow[r, "f" swap]
        \arrow[rrd, bend right=15, "f'" swap]
        & z
        \\ & & z'
    \end{tikzcd}
\end{displaymath}
então existe um \emph{único} morfismo $\theta: z \to z'$ que satisfaça as igualdades $\theta \circ g = g'$ e $\theta \circ f = f'$, ou seja, que faça o diagrama abaixo comutar.
\begin{displaymath}
    \begin{tikzcd}
        w
        \arrow[r, "\alpha"]
        \arrow[d, "\beta" swap]
        & x
        \arrow[d, "g"]
        \arrow[rdd, bend left=15, "g'"]
        \\ y
        \arrow[r, "f" swap]
        \arrow[rrd, bend right=15, "f'" swap]
        & z
        \arrow[rd, dashed, "\theta" description]
        \\ & & z'
    \end{tikzcd}
\end{displaymath}

Por vezes, diremos também que o diagrama
\begin{equation}\label{eq:diagrama_pre-pushout}
    \begin{tikzcd}
        w
        \arrow[r, "\alpha"]
        \arrow[d, "\beta" swap]
        & x
        \\ y
    \end{tikzcd}
\end{equation}
é um \textbf{diagrama de pré-pushout}, e que a tripla $(z,g,f)$ é um pushout do diagrama \eqref{eq:diagrama_pre-pushout}.

Sendo definido por meio uma propriedade universal, pushouts satisfazem uma certa propriedade de unicidade a menos de isomorfismos.

\begin{prop}
    Suponha que os dois diagramas abaixo sejam pushouts em uma categoria $\mathsf{C}$.
    \begin{displaymath}
        \begin{tikzcd}
            w
            \arrow[r, "\alpha"]
            \arrow[d, "\beta" swap]
            & x
            \arrow[d, "g"]
            \\ y
            \arrow[r, "f" swap]
            & z
        \end{tikzcd}
        \qquad
        \begin{tikzcd}
            w
            \arrow[r, "\alpha"]
            \arrow[d, "\beta" swap]
            & x
            \arrow[d, "g'"]
            \\ y
            \arrow[r, "f'" swap]
            & z'
        \end{tikzcd}
    \end{displaymath}
    Então existe um \emph{único} isomorfismo $\theta: z \to z'$ satisfazendo as igualdades $\theta \circ g = g'$ e $\theta \circ f = f'$.
\end{prop}

\begin{proof}
    Começamos com uma observação simples mas de extrema importância.
    Note que o morfismo $\id_z: z \to z$ faz o diagrama abaixo comutar.
    \begin{displaymath}
        \begin{tikzcd}
            w
            \arrow[r, "\alpha"]
            \arrow[d, "\beta" swap]
            & x
            \arrow[d, "g"]
            \arrow[rdd, bend left=15, "g"]
            \\ y
            \arrow[r, "f" swap]
            \arrow[rrd, bend right=15, "f" swap]
            & z
            \arrow[rd, "\id_z" description]
            \\ & & z
        \end{tikzcd}
    \end{displaymath}
    A propriedade universal do pushout no entanto garante que $\id_z$ é na verdade o \emph{único} morfismo do tipo $z \to z$ que faz tal diagrama comutar.
    Assim, se $\varphi: z \to z$ é um morfismo que satisfaz as igualdades $\varphi \circ g = g$ e $\varphi \circ f = f$, então necessariamente devemos ter $\varphi = \id_z$.

    Vamos agora obter o isomorfismo em questão.
    A propriedade universal do pushout garante a existência de um único morfismo $\theta: z \to z'$ fazendo o diagrama abaixo comutar.
    \begin{displaymath}
        \begin{tikzcd}
            w
            \arrow[r, "\alpha"]
            \arrow[d, "\beta" swap]
            & x
            \arrow[d, "g"]
            \arrow[rdd, bend left=15, "g'"]
            \\ y
            \arrow[r, "f" swap]
            \arrow[rrd, bend right=15, "f'" swap]
            & z
            \arrow[rd, dashed, "\theta" description]
            \\ & & z'
        \end{tikzcd}
    \end{displaymath}
    
    Veja que $\theta$ satisfaz as duas igualdades impostas pelo enunciado.
    Resta então mostrarmos que esse morfismo é na verdade um isomorfismo, e para isso vamos exibir explicitamente o morfismo inverso.
    Usando novamente a propriedade universal do pushout obtemos o único morfismo $\theta': z' \to z$ fazendo o diagram abaixo comutar.
    \begin{displaymath}
        \begin{tikzcd}
            w
            \arrow[r, "\alpha"]
            \arrow[d, "\beta" swap]
            & x
            \arrow[d, "g'"]
            \arrow[rdd, bend left=15, "g"]
            \\ y
            \arrow[r, "f'" swap]
            \arrow[rrd, bend right=15, "f" swap]
            & z'
            \arrow[rd, dashed, "\theta'" description]
            \\ & & z
        \end{tikzcd}
    \end{displaymath}

    Vamos mostrar que $\theta$ e $\theta'$ são inversos.
    O truque para mostrarmos a igualdade $\theta' \circ \theta = \id_z$ é usar a caracterização do morfismo idêntico $\id_z$ dada no primeiro parágrafo.
    Por um lado,
    \begin{displaymath}
        \theta' \circ \theta \circ f
        = \theta' \circ f' = f,
    \end{displaymath}
    e analogamente,
    \begin{displaymath}
        \theta' \circ \theta \circ g
        = \theta' \circ g'
        = g.
    \end{displaymath}S
    Segue então da conclusao do primeiro parágrafo que $\theta' \circ \theta = \id_z$.
    A demonstração da validade da igualdade $\theta \circ \theta' = \id_{z'}$ é similar, pois, analogamente ao que ocorre com $\id_z$, o morfismo $\id_{z'}$ é caracterizado unicamente por satisfazer as igualdades $\id_{z'} \circ g' = g'$ e $\id_{z'} \circ f' = f'$.
    Assim, as sequências de igualdades
    \begin{displaymath}
        \theta \circ \theta' \circ g'
        = \theta \circ g
        = g'
    \end{displaymath}
    e
    \begin{displaymath}
        \theta \circ \theta' \circ f'
        = \theta \circ f
        = f'
    \end{displaymath}
    implicam imediatamente a igualdade desejada.
\end{proof}

Veremos diversos exemplos concretos de pushouts ao longo do texto, mas desde já podemos estudar um exemplo geral que pode ser especializado para diferentes contextos.

\begin{exem}
    Seja $\mathsf{C}$ uma categoria contendo um objeto inicial $0 \in \mathsf{C}$.
    Dados dois objetos quaisquer $x,\, y \in \mathsf{C}$, temos os morfismos únicos $!_x: 0 \to x$ e $!_y: 0 \to y$ associados ao objeto inicial em questão.
    Suponha que o diagrama de pré-pushout
    \begin{displaymath}
        \begin{tikzcd}
            0
            \arrow[r, "!_x"]
            \arrow[d, "!_y" swap]
            & x
            \\ y
        \end{tikzcd}
    \end{displaymath}
    admita um pushout como abaixo.
    \begin{displaymath}
        \begin{tikzcd}
            0
            \arrow[r, "!_x"]
            \arrow[d, "!_y" swap]
            & x
            \arrow[d, "i"]
            \\ y
            \arrow[r, "j" swap]
            & z
        \end{tikzcd}
    \end{displaymath}

    Afirmamos que nesse caso $z$ é um \emph{coproduto} dos objetos $x$ e $y$, sendo $i$ e $j$ as injeções canônicas.
    Podemos mostrar isso simplesmente verificando que a propriedade universal do coproduto é satisfeita.
    Seja $z'$ um outro objeto de $\mathsf{C}$ equipado com morfismos $i': x \to z'$ e $j': y \to z'$.
    Como existe um único morfismo do tipo $0 \to z'$, os morfismos compostos $i' \circ !_x,\, j' \circ j' \circ !_y: 0 \to z'$ são iguais, ou seja, o ``quadrado externo'' no diagrama abaixo é comutativo.
    A propriedade universal do coproduto implica então a existência de um único morfismo $h:z \to z'$ fazendo o diagrama todo comutar.
    \begin{displaymath}
        \begin{tikzcd}
            0
            \arrow[r, "!_x"]
            \arrow[d, "!_y" swap]
            & x
            \arrow[d, "i"]
            \arrow[rdd, bend left=15, "i'"]
            \\ y
            \arrow[r, "j" swap]
            \arrow[rrd, bend right=15, "j'" swap]
            & z
            \arrow[rd, dashed, "h" description]
            \\ & & z'
        \end{tikzcd}
    \end{displaymath}
    Em outras palavras, os morfismos $i'$ e $j'$ podem ser unicamente fatorados pelos morfismos $i$ e $j$, respectivamente, o que significa precisamente que a triple $(z,i,j)$ é um coproduto dos objetos $x$ e $y$.

    Esse exemplo mostra que, em uma categoria que tenha um objeto inicial, coprodutos podem ser vistos como pushouts.
    Dito de outra forma, um pushout pode ser visto como uma generalização de um coproduto.
    Mais adiante, veremos inclusive que, em categorias suficientemente ricas, pushouts podem ser construídos \emph{a partir} de coprodutos usando um certo tipo de quociente.
\end{exem}

\subsection{Propriedades gerais}

O objetivo dessa seção é demonstrar algumas propriedades de pushouts que são válidas em qualquer categoria.
Posteriormente, estudaremos pushouts em exemplos particulares de categorias e teremos a oportunidade de aplicar tais resultados a exemplos concretos.

O primeiro resultado oferece uma receita para a construção de pushouts em categorias suficientemente ricas relacionando-o com coprodutos e coequalizadores.

\begin{teo}
   Considere o diagrama abaixo em uma categoria $\mathsf{C}$.
   \begin{displaymath}
    \begin{tikzcd}
        w
        \arrow[r, "\alpha"]
        \arrow[d, "\beta" swap]
        & x
        \\ y
    \end{tikzcd}
   \end{displaymath}
   Suponha que os objetos $x$ e $y$ admitam um coproduto $x+y$ com injeções canônicas $i_1: x \to x + y$ e $i_2: y \to x+y$.
   Nesse contexto, o diagrama acima admite um pushout se, e somente se, o par de morfismos paralelos
   \begin{displaymath}
    \begin{tikzcd}
        w
        \arrow[r, shift left=1.25, "i_1 \circ \alpha"]
        \arrow[r, shift right=1.25, "i_2 \circ \beta" swap]
        & x+y
    \end{tikzcd}
   \end{displaymath}
   admite um coequalizador.
\end{teo}

\begin{proof}
    Suponha inicialmente que exista um pushout como abaixo.
    \begin{displaymath}
        \begin{tikzcd}
            w
            \arrow[r, "\alpha"]
            \arrow[d, "\beta" swap]
            & x
            \arrow[d, "g"]
            \\ y
            \arrow[r, "f" swap]
            & z
        \end{tikzcd}
    \end{displaymath}
    Os morfismos $g$ e $f$ induzem um morfismo $\langle g,f \rangle: x +y \to z$ por meio da propriedade universal do coproduto.
    Vamos mostrar que esse morfismo induzido é o coequalizador desejado.
    Note, em primeiro lugar, que ele coequaliza os morfismos necessários, pois
    \begin{align*}
        \langle g,f \rangle \circ i_1 \circ \alpha
        & = g \circ \alpha \\
        & = f \circ \beta \\
        & = \langle g,f \rangle \circ i_2 \circ \beta.
    \end{align*}
    Em outras palavras, temos o diagrama comutativo abaixo.
    \begin{displaymath}
        \begin{tikzcd}
            w
            \arrow[r, shift left=1.25, "i_1 \circ \alpha"]
            \arrow[r, shift right=1.25, "i_2 \circ \beta" swap]
            & x+y
            \arrow[r, "{\langle g,f \rangle}"]
            & z
        \end{tikzcd}
    \end{displaymath}

    Precisamos mostrar que, se $p: x+y \to z'$ é outro morfismo coequalizando $i_1 \circ \alpha$ e $i_2 \circ \beta$, então existe um único morfismo $h: z \to z'$ tal que $h \circ \langle g,f \rangle = p$.
    \begin{displaymath}
        \begin{tikzcd}
            w
            \arrow[r, shift left=1.25, "i_1 \circ \alpha"]
            \arrow[r, shift right=1.25, "i_2 \circ \beta" swap]
            & x+y
            \arrow[r, "{\langle g,f \rangle}"]
            \arrow[rd, "p" swap]
            & z
            \arrow[d, dashed, "h"]
            \\ & & z'
        \end{tikzcd}
    \end{displaymath}
    Note que, como os morfismos $p$ e $h \circ \langle g,f \rangle$ têm como domínio o coproduto $x+y$, a propriedade universal do produto garante a seguinte equaivalência:
    \begin{displaymath}
        p = h \circ \langle g,f \rangle
        \iff
        \begin{cases}
            p \circ i_1 = h \circ \langle g,f \rangle \circ i_1 \\
            p \circ i_2 = h \circ \langle g,f \rangle \circ i_2.
        \end{cases}
    \end{displaymath}
    Usando as igualdades que definem o morfismo induzido $\langle g,f \rangle$, a equivalência acima pode ser reescrita da seguinte maneira:
    \begin{displaymath}
        p = h \circ \langle g,f \rangle \iff
        \begin{cases}
            p \circ i_1 = h \circ g \\
            p \circ i_2 = h \circ f.
        \end{cases}
    \end{displaymath}
    
    
    Vamos então construir o morfismo desejado.
    Como por hipótese $p$ satisfaz a igualdade $p \circ i_1 \circ \alpha = p \circ i_2 \circ \beta$, o ``quadrado exterior'' no diagrama abaixo é comutativo, portanto a propriedade universal do pushout garante a existência de um único morfismo $h: z \to z'$ fazendo o diagrama todo comutar.
    \begin{displaymath}
        \begin{tikzcd}
            w
            \arrow[r, "\alpha"]
            \arrow[d, "\beta" swap]
            & x
            \arrow[d, "g"]
            \arrow[rdd, bend left=15, "p \circ i_1"]
            \\ y
            \arrow[r, "f" swap]
            \arrow[rrd, bend right=15, "p \circ i_2" swap]
            & z
            \arrow[rd, dashed, "h" description]
            \\ & & z'
        \end{tikzcd}
    \end{displaymath}
    Segue da comutatividade do diagrama que esse único morfismo $h$ é exatamente o morfismo procurado.

    Reciprocamente, suponha agora que os morfismos $i_1 \circ \alpha,\, i_2 \circ \beta: w \to x+y$ admitam um coequalizador $p: x+y \to z$.
    Vamos mostrar então que o diagrama abaixo é um pushout.
    \begin{displaymath}
        \begin{tikzcd}
            w
            \arrow[r, "\alpha"]
            \arrow[d, "\beta" swap]
            & x
            \arrow[d, "p \circ i_1"]
            \\ y
            \arrow[r, "p \circ i_2" swap]
            & z
        \end{tikzcd}
    \end{displaymath}
    Seja $z'$ um outro objeto juntamente com morfismos $j_1: x \to z'$ e $j_2: y \to z'$ satisfazendo a igualdade $j_1 \circ \alpha = j_2 \circ \beta$.
    Isso garante que o morfismo $\langle j_1,j_2 \rangle: x+y \to z'$ induzido pela propriedade universal do coproduto coequaliza os morfismos paralelos $i_1 \circ \alpha$ e $i_2 \circ \beta$; portanto a propriedade universal do coequalizador garante a existência de um único morfismo $h: z \to z'$ fazendo o diagram abaixo comutar.
    \begin{displaymath}
        \begin{tikzcd}
            w
            \arrow[r, shift left=1.25, "i_1 \circ \alpha"]
            \arrow[r, shift right=1.25, "i_2 \circ \beta" swap]
            & x+y
            \arrow[r, "p"]
            \arrow[rd, "{\langle j_1,j_2 \rangle}" swap]
            & z
            \arrow[d, dashed, "h"]
            \\ & & z'
        \end{tikzcd}
    \end{displaymath}
    Vemos então que
    \begin{displaymath}
        h \circ p \circ i_1
        = \langle j_1,j_2 \rangle \circ i_1
        = i_1,
    \end{displaymath}
    e analgoamente também podemos mostrar que $h \circ p \circ i_2 = j_2$; portanto $h$ é exatamente o morfismo desejado.
\end{proof}

\end{document}